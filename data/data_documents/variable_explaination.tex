\documentclass[11pt,a4paper]{article}
\usepackage[utf8]{inputenc}
\usepackage[english]{babel}
\usepackage{geometry}
\usepackage{booktabs}
\usepackage{longtable}
\usepackage{array}
\usepackage{multirow}
\usepackage{xcolor}
\usepackage{colortbl}
\usepackage{graphicx}

\geometry{margin=1in}

\title{Data Structure Documentation}
\author{Your Name}
\date{\today}

\begin{document}

\maketitle

\section{Overview/Background}
There are \textcolor{blue}{3} seperate data files(table formats) store the three types of data respectively:
\begin{itemize}
    \item \textbf{Environmental Data}
    \item \textbf{Taxa Data}
    \item \textbf{Stressor Data}
\end{itemize}

These original data files carry more columns that this project needs, some information are structured in an unpleasant way and 
some columns are repeated across the three. \textbf{To keep the original integrity and track of changes,
I keep the original files and produce new files the wanted data structure from them.}

The motivation and producing process across the three original files are talked as follows.

\section{Environmental data}

\subsection{Raw data structure}
The originally complete environmental data set has many columns, they can be categorized into the \textcolor{blue}{3}
types of information for each sampled site:
\begin{enumerate}
    \item {\large \textbf{Sample identifier/Sampling information}}
    \begin{enumerate}
        \item \textbf{Integrated Code}: A tidy unique identifier for each sampled site with the naming of "waterbody + number". But it does not exist in other two files.
        \item \textbf{Site Name(all sources)}: The initial identifier from the three surveys, there are no standard naming rules across the surverys, and there are 
        some missing values.
        \item \textbf{Station NO ID(Zhang)}: The exact same identifier as the "Site Name(all sources)" in all non-NA entries, compiled by Zhang.
        There are also some missing values 
        in it but it complements all the missing values in "Site Name(all sources)".
        \footnote{It is reasonable to assume the missing values in it are deliberate for 
        ease of analysis.}
        \item \textbf{Waterbody}: The waterbody each site belongs to, there are 3 in total: Detroit River(DR), St. Clair River(SCR), and Lake St. Clair(LSC).
        \item \textbf{Year sampled}: The year each sample was collected. There are more than 3 years of the surverys, minority of them were from 2005.
        \item \textbf{* correspondence}: These '* correspondence' like columns are binary indicators to indicate if there are "*" relevant values missing in the 
        row. They are not needed in this analysis and the missing in any type of variables will be detected and handled in later coding steps.
        \item \textbf{* vs *}: These "* vs *" like columns are binary indicators, they indicate if the associated taxa or stressor data
        of the same site(Site Name or Station NO ID) are completely available. 
        These columns are also not needed here and this issue will be solved by the inner-join operation and checking the dropped rows in the merged dataframe.
    \end{enumerate}
    \item {\large \textbf{Environmental Variables:}}
    \begin{enumerate}
        \item \textbf{Latitude}: Latitude of the sampled site.
        \item \textbf{Longitude}: Longitude of the sampled site.
        \item \textbf{LOI/Total Organic Carbon (\%)}: Loss in Ignition, a measure of organic matter content in the sediment. It sometimes is named as Total Organic Carbon.
        \item \textbf{Measured Depth (m)}: The depth of the water column at the sampled site. 
        \item \textbf{Temperature (C)}: The water temperature at the sampled site in Celsius.
        \item \textbf{Water Do Bottom (mg/L)}: The dissolved oxygen at the bottom of the water column at the sampled site, in mg/L.
        \item \textbf{MPS (Phi)}: Mean Particle Size of the sediment at the sampled site, in Phi.
        \item \textbf{Velocity at bottom (m/s)\footnote{Only samples from Detroit River have this estimated measurement, but not all of DR samples 
        have this measurement.}}: The water velocity at the bottom of the water column at the sampled site, in m/s.
        \item \textbf{L *}: The variables with names starting with "L" are the logorithmically transformed variables of the original environmental variables *.
        The base of the logorithm is 10, except the \textit{LMPS} taking base 2. To avoid issues with log(0), 1 is added to * before the transformation.
        \[L* = log_{10}(* + 1) \quad LMPS = log_{2}(MPS + 1 )\]
    \end{enumerate}

    \item {\large \textbf{Clustering labels of reference sites}}
    \begin{enumerate}
        \item \textbf{Clusters:} The cluster labels(integers) of the selected reference sites in the Detroit River case study. 
        The specific values correspond to the clusters that the reference sites belong to. The missing values indicate the non-reference sites.
        \item \textbf{RelMax RefSite:} A binary indicator to indicate if the site is a reference site or not, at a higher level than the cluster labels.
    \end{enumerate}
\end{enumerate}

\subsection{Produced data structure from the raw environmental data}
As talked above, not all columns in the raw data are needed in this analysis. The wanted information should include:
\begin{enumerate}
    \item A commonly existing unique identifier across the three data files.
    \item The original environmental variables and the log-transformation rules(the log-transformed values are not necessary).
    \item The previous clustering labels to be used as a comparison basis.
\end{enumerate}

\begin{table}[h!]
\centering
\caption{Produced and kept columns from the raw environmental data}
\begin{tabular}{|>{\centering\arraybackslash}m{3cm}|p{10cm}|}
\hline
\rowcolor{gray!30}
\textbf{Variable Name} & \textbf{Description} \\
\hline
StationID & Unique identifier from the union of the two columns: Site Names (all sources) and Station NO ID (Zhang) \\
\hline
Year & Original Year Sampled column \\
\hline
Waterbody & The waterbody column but simplified into three capital letters: DR, LSC, SCR \\
\hline
Latitude & Latitude of the sampled site \\
\hline
Longitude & Longitude of the sampled site \\
\hline
LOI & The original LOI variable in the raw data. \\
\hline
Measured Depth & The original Measured Depth variable in the raw data. \\
\hline
Temperature & The original Temperature variable in the raw data. \\
\hline
Water DO Bottom & The original Water DO Bottom variable in the raw data. \\
\hline
MPS & The original MPS variable in the raw data. \\
\hline
Velocity at bottom & The original Velocity at bottom variable in the raw data. \\
\hline
clusters & The original cluster labels in the raw data for Detroit River study, indicating the cluster labels for reference sites. \\
\hline
RelMax RefSite & The original RelMax RefSite variable in the raw data, indicating if a site is a reference site or not. \\
\hline
\end{tabular}
\end{table}

\begin{figure}[!h]
    \centering
    \includegraphics[width=1\textwidth]{images/shot_of_env_data.png}
    \caption{Produced environmental data structure}
    \label{fig:shot_of_env_data}
\end{figure}

\section{Taxa data}

\subsection{Raw data structure}
The original taxa data has a clearer structure than the environmental data.
It has the same columns: integrated Code, Site Names(all sources), Station NO ID(Zhang), Waterbody, 
Latitude, Longitude, Year Sampled. These columns are the same as the ones in the environmental data.
The other columns are the 16 taxa variables and two extra columns: Validation Code and Site.

These repeated columns are not shown in this section, the other columns are:
\begin{enumerate}
    \item \textbf{Sampling information* (\(\times 7\))}: the same first 7 columns as the environmental data, showing 
    the original and compiled identifiers, waterbody, location and year information.
    \item \textbf{Taxa*(\(\times 16\))}: The 16 taxa variables, they are the abundances of the 16 benthic invertebrate taxa groups.
    \item \textbf{Validation Code}: A binary indicator to show if the associated data exists in other two
    data files for this site (identified by Site Name or Station NO ID).
    \item \textbf{Site}: The union of the two columns: Site Names (all sources) and Station NO ID (Zhang),
    as a unique identifier.
\end{enumerate}

On top of the original taxa data, there are \textcolor{blue}{5} sub-tables showing the clustering results for two level studies:
Corridor-wide and Detroit River. The first \textcolor{blue}{2} sub-tables are the clustering results(2 clusters) for the Corridor-wide study,
containing all 311 sites in the taxa data.
The rest \textcolor{blue}{3} sub-tables are the clustering results(3 cluster) for the Detroit River study,
containing 213 sites in the taxa data.

Therefore, the first two sub-tables are concated vertically to produce a clustering result for all 311 sites for 
corridor-wide study. The last three sub-tables are concated vertically
to produce a clustering result for all 213 sites for Detroit River study.

\begin{enumerate}
    \item \textbf{Corridor-wide cluster}: The clustering results for all 311 sites across the corridor,
    2 clusters in total.
    \item \textbf{Detroit River cluster}: The clustering results for all 213 sites in Detroit River, 
    3 clusters in total.
\end{enumerate}

\subsection{Produced data structure from the raw taxa data}
The sampling information columns contain the same information as the same columns in environmental data,
but the unique identifier - StationID - is needed for the merging operation later. 
Therefore, the wanted columns should include: 

\begin{table}[h!]
\centering
\caption{Kept columns from the raw taxa data}
\begin{tabular}{|>{\centering\arraybackslash}m{4cm}|p{11cm}|}
\hline
\rowcolor{gray!30}
\textbf{Variable Name} & \textbf{Description} \\
\hline
StationID & Unique identifier from the union of the two columns: Site Names (all sources) and Station NO ID (Zhang) \\
\hline
Year & Original Year Sampled column from taxa data. To a site existing in both files, its value equal to the 
the year value from another data file. \\
\hline
Waterbody & The waterbody column but simplified into three capital letters: DR, LSC, SCR. Same value for the site that exists in both files.\\
\hline
Latitude & Latitude of the sampled site. Same value for the site that exists in both files.\\
\hline
Longitude & Longitude of the sampled site. Same value for the site that exists in both files.\\
\hline
taxa variables(\(\times 16\))& The original taxa variables in the raw data, indicating the abundances of the 16 benthic invertebrate taxa groups. \\
\hline
Corridor clusters & The clustering results for all 311 sites across the corridor, 2 clusters in total. \\
\hline
Detroit clusters & The clustering results for all 213 sites in Detroit River, 3 clusters in total. \\
\hline
\end{tabular}
\end{table}

After the cleaning operation, the produced taxa data structure is shown in Figure \textcolor{blue}{\ref{fig:shot_of_taxa_data}}.

\begin{figure}[!h]
    \centering
    \includegraphics[width=1\textwidth]{images/shot_taxa_all.png}
    \caption{Produced taxa data structure from the raw data.
    (There are resting columns not shown, 16 taxa variables/columns in total)}
    \label{fig:shot_of_taxa_data}
    \end{figure}

The clustering results for the two studies: corridor-wide and Detroit River, are 
stored in two seperate tables with StationID as index for easy merging later.
These tables of clustering results are shown in
Figure \textcolor{blue}{\ref{fig:shot_of_taxa_corr_clusters}} and \textcolor{blue}{\ref{fig:shot_of_taxa_DR_clusters}}.

\begin{figure}[!h]
    \centering
    \begin{minipage}{0.2\textwidth}
        \centering
        \includegraphics[width=\textwidth]{images/shot_taxa_corr_clusters.png}
        \caption{Clustering results for all 311 sites across the corridor(snapshot)}
        \label{fig:shot_of_taxa_corr_clusters}
    \end{minipage}
    % \hfill
    \hspace{2cm}
    \begin{minipage}{0.2\textwidth}
        \centering
        \includegraphics[width=\textwidth]{images/shot_taxa_DR_clusters.png}
        \caption{Clustering results for all 213 sites in Detroit River(snapshot)}
        \label{fig:shot_of_taxa_DR_clusters}
    \end{minipage}
\end{figure}


\section{Merged environmental and taxa data}
At this stage, the produced environmental data and taxa data share the same unique identifier - StationID.
An inner-join operation is performed on the StationID column to combine the two datasets, producing a 
complete dataset of sites with both types of data available. This join operation also reveals the 
sites that do not have associated data in either of the two datasets, which are dropped in the final merged dataset.

As the data operation logic talked in the proposal, the conceptually merging process and the merged 
data structure are like:

\begin{figure}[!h]
    \centering
    \includegraphics[width=.8\textwidth]{images/data_operation_1.png}
    \caption{Idealization: how the cleaned data sets are merged together and the resulting data structure.}
    \label{fig:data_operation_1}

\end{figure}

\begin{figure}[!h]
    \centering
    \includegraphics[width=.8\textwidth]{images/merged_env_taxa_data.png}
    \caption{Practically: produced merged environmental and taxa data structure.
    (The multi-level column indices are highlighted in colors; It is a snapshot of first columns, the 
    clustering result columns remained at the right end of the table.)}
    \label{fig:merged_env_taxa_data}
\end{figure}

\section{The purpose of maintaining datasets in a wider table with multi-level column indices}
There are three main reasons for maintaining the three original datasets in a wider table with multi-level column indices,
they are listed as the order of importance as follows:

\begin{enumerate}
    \item \textbf{Avoiding to create and spread many data objects in both coding and storing stages.}
    
    The framework of this project is designed to be computationally demanding, many itermediate computing results are 
    produced with numbers of testing and tuning tasks, and the results will be used for guiding the next steps. 
    If these intermediate results are stored seperately and instantiated as seperate data objects in the coding space,
    it will be hard to tract and manage them, even the naming work will be a burden.
    Additionaly, many test and tuning tasks need to bundle the data and intermediate results, it requires
    the aligning and matching operations across these objects if they were stored separately,
    very error-prone and inefficient.

    Therefore, maintaining only the core data and the conclusive intermediate results in a single data object is
    good for both coding work and data management.

    \item \textbf{A faster way to do column-wise operation across all types of data.}
    
    Keeping all important data in one data object makes the column-wise operations broadcasting 
    to all columns across all types of data easy. For example, assuming the pollution scores have been stored
    for all sites in the merged data structure, the pollution-relevant grouping operations will 
    automatically group the sites with both environmental and taxa data. It saves the time for the process:
    \textit{grouping on pollution scores \(\rightarrow\) group indexing on environmental and taxa data  \(\rightarrow\) computing on each group.}

    \item \textbf{Easier to track, read and inspire ideas from the data.}
    
    Keeping columns from all types of data together produces a larger(or complete) matrix of data, 
    a more comprehensive view of the data.
    It is easier to read and understand the data and structure by leveraging the visually clear table format.
    This sub-block and complete view of the data also inspires ideas for analysis and modeling.
\end{enumerate}

\begin{figure}[!h]
    \centering
    \includegraphics[width=.8\textwidth]{images/merged_env_taxa_data.png}
    \caption{An instance of the merged data structure with multi-level column indices that meet the above purposes.}
    \label{fig:merged_env_taxa_data}
\end{figure}

\section{More to come later: Stressor data and the final complete data set.}

\end{document}