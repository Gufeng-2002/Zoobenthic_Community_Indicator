\chapter{Taxa Assemblages Cluster Analysis}


\textcolor{red}{There should be 2 to 4 articles contain the major methdology used in this section,
either statistically and ecologically.}

\textcolor{red}{The work needs to build upon that framework of methodology.}



\begin{figure}[!h]
    \centering
    \includegraphics[width=0.9\textwidth]{images/demo_cluster.png}
    \caption{Inputs and outputs at this stage}
\end{figure}


\section{Introduction}
Put the literature review of cluster analysis in identifying zoobenthic assemblages here. 
List supports and reasons for using hierarchical cluster analysis method.

\section{Waiting to seperate}
The major steps in zoobenthic assemblages cluster analysis are summarized as follows:
\begin{enumerate}
    \item Explain the meanings of taxa variables; 
    the individual level taxa and family level taxa variables (similar like chemical variables);
    the misalignment of variables across three surveys;

    \item Describe the statistical properties of the taxa data;
    taking the motivation and reasons to pre-process the data both analytically and ecologically;
    then transform the data as needed (octave scale previously);
    \item Ward's method of cluster analysis with Manhattan distance measure is applied to 
    do hierarchical clustering on the taxa data;
    \item \textcolor{red}{why to take this specific clustering method and the 
    distance measure here;}
    \item Using ANOVA-like F-ratios, identify the different importance of zoobenthic taxa in distinguishing sites into clusters;
    \item \textcolor{red}{how to interpret this clustering results of relation with environmental features,
    and this might also work in interpreting the ANOVA results;}
    \item \textcolor{red}{How to evaluate the clustering results
    rather than just accepting the results with a wanted number of clusters;}
\end{enumerate}

\section{Conclusions}

Interpret the clustering results and performances here, 
illustrate its meaning and connections with the following DFA analysis.
\textcolor{red}{The major discussion here should be around the relationship
between taxa assemblages and environmental/habitat features.}