
\chapter{Zoobenthic Condition Index by Ordination Methods}

\textcolor{red}{There should be 2 to 4 articles contain the major methdology used in this section,
either statistically and ecologically.}

\textcolor{red}{The work needs to build upon that framework of methodology.}


\begin{figure}
    \centering
    \includegraphics[width=0.9\textwidth]{images/demo_Ordination.png}
    \caption{Inputs and outputs at this stage}
\end{figure}

\section{Introduction}
Put the literature review of ordination methods in identifying zoobenthic condition here.
List supports and reasons for using Bray-Curtis Ordination method.

\section{Waiting to seperate}
The major steps in zoobenthic assemblages cluster analysis are summarized as follows:
\begin{enumerate}
    \item Explain the meanings of environmental variables;
    the availability of variables across three surveys;

    \item Describe the statistical properties of the environmental data;
    taking the motivation and reasons to pre-process the data both analytically and ecologically;
    then transform the data as needed (log1p scale previously);

\end{enumerate}
