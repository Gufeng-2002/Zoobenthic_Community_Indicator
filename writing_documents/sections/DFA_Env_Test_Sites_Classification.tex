\chapter{DFA of habitat features for Taxa Assemblage Clusters of Test Sites}


\textcolor{red}{There should be 2 to 4 articles contain the major methdology used in this section,
either statistically and ecologically.}

\textcolor{red}{The work needs to build upon that framework of methodology.}


\begin{figure}[!h]
    \centering
    \includegraphics[width=0.9\textwidth]{images/demo_DFA.png}
    \caption{Inputs and outputs at this stage}
\end{figure}


\section{Introduction}
Put the literature review of DFA in classifying environmental/habitat features here.
To this classification task, compare with other classification methods, like logistic regression,
support vector machine, etc. 
List supports and reasons for using discriminant function analysis method.

\section{Waiting to seperate}
The major steps in zoobenthic assemblages cluster analysis are summarized as follows:
\begin{enumerate}
    \item Explain the meanings of environmental variables;
    the availability of variables across three surveys;

    \item Describe the statistical properties of the environmental data;
    taking the motivation and reasons to pre-process the data both analytically and ecologically;
    then transform the data as needed (log1p scale previously);

    \item Build a forward stepwise DFA model of habitat features for the taxa assemblages clusters.
    Identify the important habitat features and 
    classify the test sites into most similar taxa assemblages clusters with the built DFA model;
    \item \textcolor{red}{why to take this specific clustering method and the 
    distance measure here;}
    \item \textcolor{red}{It needs to be model validation for the DFA here, like cross-validation;
    However, there is no true labels for training data because labels are chosen from clustering results;
    it requires a solid discussion on the reliability of the clustering results as good labels for building the DFA model;}
    \item \textcolor{red}{Another concern is the small sample size of labeled data, cause the 
    reference sites are only around 20\% of the total sites; it might be a good question to solve;}
\end{enumerate}


\section{Conclusions}
